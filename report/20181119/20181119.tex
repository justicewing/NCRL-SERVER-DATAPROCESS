%!TEX program = xelatex
%%%%%%%%%%%%%%%%%%%%%%%这是导言部分的开始%%%%%%%%

%========= 导言部分声明文档的类型=================
\documentclass{article}

%=========导言部分可可以加载宏包=================
\usepackage{amsmath}                % 数学公式排版宏包
\usepackage{amssymb}                % 数学符号命令宏包
\usepackage{amsthm}                 % 数学定理宏包
\usepackage[UTF8]{ctex}             % 中文输入宏包
\usepackage[a4paper]{geometry}      % 页面设置宏包
\usepackage{setspace}               % 行间距宏包
\usepackage{graphicx}               % 图片宏包
\usepackage{listings}               % 代码宏包
\usepackage{color}					% 颜色宏包
\usepackage{xcolor}                 % 颜色处理宏包
\usepackage{float}                  % 浮动对象式样宏包
\usepackage{fontspec}
\usepackage{enumerate}				% 列举编号包

%=========页面设置==============================
\geometry{left=1cm,right=1cm,top=1cm,bottom=2cm}
\onehalfspacing
\setlength\parindent{0em}

%=========代码格式设置============================
\definecolor{dkgreen}{rgb}{0,0.6,0}
\definecolor{gray}{rgb}{0.5,0.5,0.5}
\definecolor{mauve}{rgb}{0.58,0,0.82}
% \setmonofont{Consolas}
\lstset{
	numbers = left, 	
	numberstyle = \color{gray}, 
	keywordstyle = \color{blue},
	commentstyle = \color{dkgreen}, 
	stringstyle = \color{mauve},
	basicstyle = \ttfamily,
	breaklines = true,
	frame = shadowbox, % 阴影效果
	rulesepcolor = \color{ red!20!green!20!blue!20} ,
	escapeinside = ``, % 英文分号中可写入中文
	xleftmargin = 2em,xrightmargin=2em, aboveskip=1em,
	framexleftmargin = 2em
} 

%=========导言部分可以定义标题信息===============
\title{组会报告}
\author{徐益}
\date{\today}
%%%%%%%%%%%%%%%%%%%%%%%这是导言部分的结束%%%%%%%%%

%%%%%%%%%%%%%%%%%%%%%%%这是正文部分的开始%%%%%%%%%
\begin{document}

%=========生成标题================================
\maketitle

%=========开始正文的输入==========================

%===========第一节=================
\section{工作内容} 
1. Debug连调程序并测试;

2. 阅读协议并修改资源映射结构。

3. 根据新的资源映射结构修改程序

%===========第二节=================
\section{Debug连调程序并测试}
\subsection{Debug结果}
MAC层接收端读参数标识早于PHY层接收端写参数标识\\
最终导致读写冲突。
\subsection{HARQ测试结果}
\begin{figure}[H]
	\centering
	\includegraphics[width = .6\textwidth]{compare.png}
	\caption{HARQ测试结果}
\end{figure}

%===========第三节=================
\section{阅读协议并修改程序结构}
\subsection{正确的层映射关系}
\begin{figure}[H]
	\centering
	\includegraphics[width = .9\textwidth]{table.png}
	\caption{Codeword-to-layer mapping for spatial multiplexing}
\end{figure}
\subsection{程序结构修改情况}
\begin{figure}[H]
	\centering
	\includegraphics[width = .7\textwidth]{str1.pdf}
	\caption{原结构}
\end{figure}
\begin{figure}[H]
	\centering
	\includegraphics[width = .55\textwidth]{str2.pdf}
	\caption{协议要求结构}
\end{figure}
\begin{figure}[H]
	\centering
	\includegraphics[width = .7\textwidth]{str3.pdf}
	\caption{修改后结果}
\end{figure}

%===========第四节=================
\section{根据新的资源映射结构修改程序}
\subsection{用多线程数据结构编写单线程程序}
\begin{figure}[H]
	\centering
	\includegraphics[width = \textwidth]{tips.png}
	\caption{发送端的单线程模块}
\end{figure}
\begin{figure}[H]
	\centering
	\includegraphics[width = \textwidth]{tips2.png}
	\caption{接收端的单线程模块}
\end{figure}

%===========第五节=================
% \section{有PRACH情况下的资源分配问题}
% \begin{figure}[H]
% 	\centering
% 	\includegraphics[width = \textwidth]{ques.pdf}
% 	\caption{两种方案}
% \end{figure}

%===========下周计划=================
% \section{下阶段计划}
% 1. 完善单线程系统(修复Bug)

\end{document}
%%%%%%%%%%%%%%%%%%%%%%%这是正文部分的结束%%%%%%%%%%%%